\section{Conclusion and Future Work}
CubanSea successfully applies Shneiderman's principles of visualization while providing a solution for topic recognition in the result space. The fuzzy-clustering algorithm together with the automated generation of topic headers facilitates users with the means to extract semantic information about the result space and to restrict their result space to those results relevant to their search. Specifically the filter functionality is very successful as a tool enabling the user to query the result space and extract subsets of it.

However, still a significant amount of issues are not dealt with. Problems were encountered during the user studies in the algorithms generating the clusters, assigning results to the clusters and generating the cluster headers. Further research is necessary in order to optimize these algorithms. It must also be evaluated whether or not the display of only the first five snippets is sufficient. Performance is obviously another factor. Different deployment methods must be examined and evaluated. Also, the prototype has to be optimized in order to achieve a highly performant application.

Since all tests were based on tasks specifically designed to compare CubanSea with Google, the results might be not transferable to real-life scenarios. Long-term testing on a broader user base is required in order to evaluate the real effectivness and quality of the system. Additionally, the reuse and analysis of selected result sets should be empowered. A prototype extension providing this was planned for this paper but dropped out of scope.