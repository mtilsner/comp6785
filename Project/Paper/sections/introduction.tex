\section{Introduction}
The primary medium used for retrieving information from the web has always been a list. In the early days of the internet, Sir Tim Berners-Lee maintained a list of all available web pages that could be used for identifying possible sources of information. Search engines replaced that list as the amount of web pages became too large to be centrally maintained. These engines, too, display the results as lists.

The downside of this list-based visualization is the limited amount of information that can be percieved and processed by the viewer simultaneously. Usually, search engines try to provide a ranking of all search results, ordering the result list accordingly. The only domain used for encoding metainformation is the horizontal position of search results inside the result list. Furthermore, in order to achieve a performance increase, the list is split into different pages that the user can browse through. This restricts both the total amount of results and the selection of possible results that can be compared directly to one another.

Search engines are a necessity for performing information retrieval on the web. Most tasks undertaken online require at some point the use of a web search engine, making them the most frequently visited pages online. Hence, the success of most of these tasks, greatly depends on the quality of those search engines. Consequently, a significant amount of research has been undertaken trying to improve their visualization technique. According to \cite{Shneiderman1996}, the basic principal for visualization is ``overview first, zoom and filter, then details on demand''. This basic principle can be considerd crutial for the success of a visualization. Unfortunately, no research undertaken so far that successfully implements this principle. It can be hypothesized that this might be one of the reasons why until now no design has been widely accepted. The primarily used web search engines Google, Yahoo, and Microsoft Live still use traditional result lists \cite{MostPopularWebSearchEngines}.

One major concern of visualization is the ambiguity of the result space. Users provide a search query that is used for identifying which results are relevant. This query, however, might refer to several distinct topics. The search term ``piracy'', for example, can refer both to modern licence transgressions regarding digital propety and to historic sea robbery. Naturally, search engines are unable to identify in which one the user is interested. Consequently, result items from both topics are being returned in shuffled manner. Traditional systems suggest that the user refines his query until he achieves a homogeneous result space unambigously identifying the desired topic. This, however, requires the user to have a preexisting knowledge about what topics are applicable to his query and to execute additional network requests.

This paper presents an alternative interface for search engines that visualizes the ambiguity of the result space while applying Shneiderman's priciple. The topics that the result space consists of are displayed, allowing the user interactive exploration.