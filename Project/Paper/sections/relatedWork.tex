\section{Related Work}
As noted above, a significant amount of research has already been undertaken in the area of search result visualization. Most projects propose designs that position result items next to each other according to their simmilarity. \cite{Rauber2000} suggests a bookshelf-like design, creating different areas for different topics. Result items are placed in accordance to their membership in these topics. The bookshelf is created on demand using a self-organized map. \cite{Paulovich2008} proposes to position search results on a two-dimensional canvas, encoding the simliarity of results in the distance between them. For this, a multi-dimensional vector is created for each result, consisting of the word frequencies of the corresponding document. This multi-dimensional vector is then be projected onto a two-dimensional plane. \cite{Einsfeld2006} uses a similar technique, projecting the vector into a three-dimensional space. Areas of result accumulation can be interpreted as distinct topics by the user. All of these techniques require the user to deduct the semantic meaning of these topics by examining the results contained. Furthermore, they position documents that belong to more than one topic in the middle between those. This, however, becomes hard to decode as the number of topics exceeds the number of dimensions available for positioning.

Another encoding technique frequently suggested is the use of connecting lines to visualize when certain documents are related or connected to one another. \cite{Zaina2005} is one of the papers suggesting this. It calculates the relationships between documents using their similarity.

A number of papers suggest using color to encode the relevance a result has to a search query. \cite{Mukherjea1999} splits the query into its different terms and assigns each a specific color. An icon is presented for every result that encodes the relevance of the result to its most relevant term using color saturation. A similar technique is proposed in \cite{Hoeber2006}. Conventional search engines use positioning to encode this information.

All of these designs have one thing in common: they either display the entire result space at once or use pagination and present the top X results to the user. The visualization presented in this paper solves this problem. It is primarily influenced by \cite{Rauber2000} and \cite{Hoeber2006} and tries to leverage their findings. Furthermore, it provides a solution for automated topic generation and aided topic recognition through the user by generating topic headers. The problem of assigning results to multiple clusters is solved by using fuzzy clustering.

Several papers discuss means of refining the the result space. \cite{Hoeber2006b} suggests aided refinement of the search query by analyzing what other terms are frequently used in the result space. This technique requires the user to resubmit his query, thus impairing the system's performance. \cite{Mukherjea1999} proposes a card design, enabling the user to select subsets of the result space. Unfortunately only one of those subsets can be displayed at a time and the subsets are automatically generated to be disjoint without giving the user any option of altering them.

A simple and more efficient approach is to filter the result space on demand using a filter query. This allows the user to refine his query and to extract relevant result documents without requiring additional network requests. The result list can simply be minimized to the subset of all results containing the filter query. Since this query can be freely defined by the user, any kind of subset can be adaptively created on demand. The design in this paper presents such a solution, integrated into the cluster-based visualization.