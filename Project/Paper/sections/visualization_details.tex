\subsection{Details On Demand: Collapsable Snippets}
The last step of Shneiderman's principle is to display details on demand. Clearly, giving all information at once is likely to overstrain the viewer. Search engines typically display snippet and url of a search result below its title. This behavior has been modified in CubanSea in order to save space and to allow the simultaneous perception of a larger amount of result items. Instead of displaying all snippets and urls, only those of the first five results are given. The use case study undertaken in this paper shows that most people typically find their answer within those items. Hence, the detailed information of the other items may be initially neglected. The user can manually choose which snippets and urls he wants to be displayed by clicking the large ``+'' sign inside the relevance square. A ``-'' hides this information respectively. Furthermore, snippets can be viewed inside tooltips by moving the mouse over the results as displayed in figure \ref{fig:details:piracy}

Beyond these details, every result can naturally be clicked in order to open the result document, just as in any other search result visualization.