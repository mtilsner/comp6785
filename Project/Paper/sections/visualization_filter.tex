\subsection{Filter: Result List Reduction}
Especially when dealing with large result lists as those likely to be yielded by omitting pagination, identification of interesting topics can be complicated. \cite{Shneiderman1996} proposes filtering the visualized data. The straightforward approach for doing so is to provide a search input field that, upon change, filters the result list using JavaScript. Since all filtering methods are executed asynchronously on the client, the user can interactively redefine filters as he likes without having to wait. 

By adding this functionality to the dynamically growing column-based result list, the user is facilitated with the means to quickly identify all results that might be of interest to him. Figure \ref{fig:filter:piracy:cari} displays the result of entering the search phrase ``cari'' into the input field. The intend of this example is to find all results somehow talking about the piracy in the caribbean. The result list is interactively reduced to those items containing the phrase ``cari''. By checking/unchecking the checkbox ``in summary'', the user can control whether the snippets shall be used for identifying relevant results or not.