\section{Perform an in-depth literature review of this topic that covers both the seminal work, as well as recent advancements. For each paper you find that is relevant to the topic, provide a brief summary of its contribution to the field.}
The following papers are also relevant. Unfortunately, I did not have the time to summarize them: \cite{Andrews2001, Berenci1999, Brin1998, Chang1999, Chirita2005, Cutting1992, Eick1994, Hearst1995, Heimonen2005, Jansen2001, Joho2004, Jones1998, Kobayashi2000, Kosala2000, Kules2006, Paek2004, Radlinski2006, Robertson1997, Sanderson1999, Spink2002, Spink2001, Sugiyama2004, Teevan2005, Vaughan2004, Wise1995, Zamir1998}

\subsection{\cite{Alonso2000}}
This paper discusses the dependability of information retrieval components and the InfoVis client. Its main concern is the visualization of large document collections in web search engines. It proposes an architecture, that allows the reuse of visualization results, thus making them accessible to multiple users. This is extremely helpful in collaboration applications and ``solves the problem on data overload on Internet''.

\subsection{\cite{Alonso2003}}
The paper tries to increase the comprehensibility of web search results by proposing a split view interface. This interface displays both the results in an ordered manner, and a short overview over selected search results, highlighting the individual search terms so that the reader can get a grasp of the relevance of the document more easily. While developing the solution, the authors concentrated primarily on performance and scalability. This is achieved by letting the server perform all visualization, thus reducing the client effort.

\subsection{\cite{Benjamin2008}}
This paper introduces a method to visualize similarities in web results, thus providing a way for detecting plagiarism. The presented solution uses Kohonen Maps to visualize the similarities, aiming for the user to be able to grasp it intuitively. It furthermore incorporates a Customized Term Weighting Scheme that extracts semantic information, thus enabling an automated classification of the search results.

\subsection{\cite{Einsfeld2006}}
This paper aims to provide a method for visualizing both content and semantic information of documents. This aim is constricted by the objective of providing an intuitive user interface. It proposes an application called \emph{DocuWorld} that visualizes documents content and meta-data together with the semantic relations between documents in a 3 dimensional canvas. It includes a method for adapting the visualization based on user input and the currently visible context.

\subsection{\cite{Hoeber2007}}
This thesis concentrates on the interactive refinement of web search terms inside existing visualizations and the exploration of results. It proposes four different tools for achieving those tasks: HotMap allows the user to add and remove search terms while exploring the result list. Furthermore, the user can define the importance of the search terms, thus allowing an interactive reordering of the result list. While HotMap is based on classic one-dimensional result lists, VisiQ provides a two dimensional displayal of the results allowing the user to easily comprehend how documents are related to one another. WordBar displays the terms most commonly used inside the documents of the result list. This list is very helpful for identifying additional terms that could be added to the query for refining the result list or that should be excluded (i.e. by using googles ``-'' prefix). The last tool presented is the Concept Clusterer, clustering documents by using color for encoding. Two user studies were conducted, showing positive results especially for the WordBar and the HotMap tool. Even though the Concept Clusterer was helpful an additional step was required including reevaluation of the results. VisiQ was not used in the studies. \\
The thesis greatly proposes the utilization of interactive tools in order to enable the user to iteratively refine an specify his search. Furthermore, it suggest a two-dimensional design space plotting the user tasks in relation to generated information about the result items at hand. In order to allow a visual representation of the search terms, a Concept Knowledge Base is introduced. Next to the benefits generated by the tools, the thesis also presents a novel evaluation method for specifically evaluating InfoVis applications for web searches.

\subsection{\cite{Konchady1998}}
This paper tries to provide a new way for visualizing search results. It concentrates on the relevance of the search results in correspondence to the query words. It displays all search results in a multi-dimensional environment (up to 3 dimensions). The user can assign the different search terms or documents to the axes. All other documents are displayed by dots with their positioning being derived from the similarity of the document with the axis value. The user can then rotate the cube and zoom in and out in order to retrieve information. Furthermore, the paper proposes a link analysis tool that extrapolates the relationship of documents between one another by looking at the links, and one visualizing the similarities of groups of people regarding their search terms.

\subsection{\cite{Weiss2001}}
This paper aims to provide an efficient way for querying documents. It uses meta-data that describes and differentiates the content of different paragraphs, a document consists of. Instead of the original document content, only the meta-data is visualized. A test experiment was run that inspired redesign the the application in order to improve the performance. A numerical advantage could not be proven. However, the paper suggests that this is due to utilization of meta-data.

\subsection{\cite{Mukherjea1999}}
This paper proposes an alternative interface for displaying the results of a web search engine, providing the user with the means to navigate through the documents based upon their relationship. It uses sets of cards that are layered in front of each other to display related documents via connecting lines. The relevance is displayed by assigning different colors to the search terms and coloring the individual result documents accordingly.

\subsection{\cite{Mukherjea1996}}
This paper proposes the visualization of search results using either a scatter point matrix, or displaying the search results in a three dimensional space. Color and dimensions of the individual document points inside the graphics are used for encoding the relevance to the search terms. No usability study is performed within the paper. Instead, one is proposed, suggesting the refinement of the findings according to the results of that study.

\subsection{\cite{Nowell1996}}
This paper discusses the visualization of library items with a tool called \emph{Envision}. Here, search result items are displayed using icons that are positioned in a matrix and using different shapes and colors for the icons. The meaning of the different means of visualization can be defined by the user. The paper discusses the results of a usability test performed on \emph{Envision}, showing positive results and user feedback. Even though the paper solely discusses the utilization for the content of a predefined library, I believe that the results are just as relevant for the area of visualizing web search results.

\subsection{\cite{Paulovich2008}}
This paper presents the \emph{Project explorer for the WWW}, an application that visualizes the results of web search engines in a multidimensional map. The positioning of the the different result items is dependent on their content. Similar documents are placed near to one another, thus allowing a preemptive processing of related document groups. Additionally, they are connected with lines, thus allowing an easy identification of related documents. The relevance to the groups specific search terms is displayed by coloring the search result items.

\subsection{\cite{Raubner2000}}
This paper suggests a redesign of web search engine interfaces oriented on classical libraries. It proposes that the results are organized by topics and displayed in a bookshelf-like manner. The meta-data of the individual search result items is to be displayed so that it can be comprehended intuitively. The language of a document found, for example, is encoded using color (i.e. German = yellow, English = blue). The actual size of the returned document is displayed by the size of the icon representing it in the library view. Since the additional effort required for generating the visualization is minimum, the authors assume the result to be effective.

\subsection{\cite{Roussinov1999}}
This paper evaluates the efficiency of making result maps used for displaying Internet search results interactive. These result maps group results together based on similarity and search term relevance. The author tries to enhance the search experience by allowing him to respecify the search terms while viewing the visualization. Furthermore, the system can generalize the grouping, thus creating fewer groups inside the result map. In a case study, this interactive features proved to be successful and were repetitively used by the users.

\subsection{\cite{Sebrechts1999}}
This paper evaluates the effectiveness of using visualization tools per se for visualizing search results. It runs a series of test with a mixed group of both professional and non-professional users, using an InfoVis tool called \emph{NIRVE}. The results of these tests are, that the best way for displaying the the information is dependent on the task at hand. While looking for a specific target is much easier when using a one-dimensional, textual visualization, clustering and categorization benefits greatly from multidimensional visualizations.

\subsection{\cite{Tvarozek2008}}
This paper proposes web search results to be displayed using of hierarchical clusters and assigning them attributes with meta-data. It suggests, that the visual presentation should also include the qualitative relationship of documents. Using positioning, connections, and coloring, the paper proposes a two dimensional visualization of the search results. In this visualization, only the most basic information should be viewed at first, enabling the user to interact with the visualization in order to explore.

\subsection{\cite{Weippl2001}}
This paper analyzes different methods for visualizing collections of hypertext, grouping them into clusters. It discusses two different two-dimensional user interfaces and one three-dimensional one. It proposes the use of a Self-Organized Map but at the same time identifies it as a bottleneck when processing high numbers of hypertexts. The authors announce further research in this area, trying to solve this issue.

\subsection{\cite{Zaina2005}}
This paper propose a two-dimensional graph displaying the results of a web search connected with lines to display relationships. The relevance of the item in regard to the search term is encoded using the result item's icon size. The semantic information can be retrieved by fetching the first 50kB for each search result. A table view displays additional non-semantic information. A case study is performed proving the solution's success to be moderate. A significant amount of the users actually preferred the table view, corresponding to a classical search result list.
