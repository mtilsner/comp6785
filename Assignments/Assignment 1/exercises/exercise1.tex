\section{Find four different definitions of �information visualization� from the existing 
literature (using proper methods for quoting and referencing). Present your own view 
and comments on each of them.}

\subsection{Preliminary thoughts}
In order to effectively examine the following definitions, information visualization has to be
evaluated for its most relevant issues. A definition should try to cover all these issues with the
minimum amount of words while at the same time being as simple and comprehensible as possible.
Also, it should be defined abstract in order to be applicable to any kind of concrete scenario.

During the lecture no January 14th, information visualization was described as "the field of study that
deals with the design and creation of visual representations of abstract data" \cite{lecture:orland}
(lecture 1, slide 3). It was furthermore described, that by providing a visual presentation, data
can be much more easily comprehended and processed by human beings (lecture 1, slide 6). Consequently,
the following issues can be defined as relevant, based on the information provided within the lecture:
\begin{description}
	\item[Scientific research] Information visualization is a field of study. It is therefor an area for
														 scientific research.
	\item[Visual]							 Obviously, information visualization is based on visual perception. It
														 concentrates on vision as the primary sense and tries to use it.
	\item[Data representation] The primary task of information visualization is to display abstract data.
	\item[Knowledge gain]			 Last, the ultimate goal of information visualization is to provide the user
														 with the means to comprehend and process the abstract data.
\end{description}
From these four issues, the \emph{scientific research} is probably the one least important. Information
visualization could just as well be understood as an area where both scientific research ans practical
application are relevant.

\subsection{Definition 1}
\begin{quotation}
	Information visualization is... \emph{"the use of computer-supported, interactive visual
	representations of data to amplify cognition."} \cite{definition:shneiderman}
\end{quotation}
This is a very concise, and generalizing definition. It covers the visual issues as well as both the
data representation and the knowledge gain. It is brief, simple, easy to comprehend and can be applied
to every kind of scenario. All requirements for a good definition are therefore met. The word "amplify"
allows information visualization to be considered as part of a greater process of cognition. It does not,
however, define information visualization as a field of study. Instead it indicates with the word "use"
information visualization to be some kind of "best practise". In my personal opinion, this is a flaw in
the definition. Furthermore, it restricts its utilization on computer aided systems, thus limiting its
applicability. It allows all kind of data to be used for representation, not just abstract data. I believe
that this is actually correct, since cognition of non-abstract data can be aided by a visual representation
just as well (for example displaying the poverty level of countries).

\subsection{Definition 2}
\begin{quotation}
  (Information) visualization is... \emph{"a graphical representation of data or concepts,"} ...that is
  either an... \emph{"internal construct of the mind"} ...or an... \emph{"external artifact supporting
  decision making."} \cite{definition:ware} (page 2)
\end{quotation}
This definition extends the object in question that is to be displayed to include concepts. Even though it
is brief, it is more difficult to comprehend. Both the scientific issue and the knowledge gain are not
touched by it. Ware only talks about \emph{supporting decision making}. Cognition in general, however, does
not necessarily require a decision to be made. The goal might simply be to extract some information that is
of interest or to discover patterns that are not as obvious in non-visual representations. The lack of cognitive
issue is a critical flaw that substantially reduces the quality of this definition. Since the definition
is intended to work for both internal and external visualizations, it does not require a computer to be present.
Ultimately, this definition covers the visualization of information in a much broader sense then the one we are
seeking for. We, however, concentrate purely on computer-based information visualization. Since the objects need
to be described by data in order to be processed by a computer, the benefit of including concepts is not relevant. Consequently, I believe that all advantages this definition provides are irrelevant to our context. Since the
definition still has disadvantages over Definition 1, it is less fitting to our context.

\subsection{Definition 3}
\begin{quotation}
  \emph{"Information visualizations attempt to efficiently map data variables onto visual dimensions in order to
  create graphic representations."'} \cite{definition:gee}
\end{quotation}
Just like Definition 2, this one does not cover the scientific nor the cognitive issue. The lack of the cognitive
issue is a critical flaw of the definition. The described mapping of data variables to visual dimensions is very
intuitive and well formulated. Since color, position, and dimension are all visual dimensions, the data can really
be understood to be mapped to those. However, there is one other non-visual dimension that is of interest for mapping
data variables: time. Visualizations can very well be animations (for example when showing the spreading of a particular disease). The chronological dimension should therefore be considered as well. Just like Definition 2, it applies to
a broader sense of visualizations, but, following the same logic, this does not create any benefit for our context. Consequently, I believe that Definition 1 is better suited for our context.

\subsection{Definition 4}
\begin{quotation}
  Information visualization is... \emph{a method of presenting data or information in non-traditional, interactive 
  graphical forms. By using 2-D or 3-D color graphics and animation, these visualizations can show the structure of
  information, allow one to navigate through it, and modify it with graphical interactions.} \cite{definition:uiuc}
\end{quotation}
This definition again does not talk about either scientific utilization or cognitive gain. Instead, it introduces
a new trait: \emph{non-traditional}. When applying on different scenarios, it becomes obvious, that \emph{non-traditional}
limits the number of appliances substantially. All scenarios where information is traditionally displayed visually
is hereby excluded (i.e. maps or signs). Instead of describing the benefit in knowledge gain, it states that 
information visualization can show the structure of information. In the terms of pattern recognition this might be
true. It does, however, not include the recognition and evaluation of proportions of non-related data sets. The
term "knowledge" is much wider then "structure" and therefore allows a broader application of a definition. Because
of that, and because of the lengthy formulation, this definition is in my opinion less efficient then the other ones.