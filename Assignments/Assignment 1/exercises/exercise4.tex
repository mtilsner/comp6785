\section{List eight distinct research topics in the field of information visualization. For
each, provide a brief description.}

\paragraph{Interactive InfoVis systems}
In order to allow the user to fully utilize visualizations, it is necessary to provide him with the means for interacting with the system. Current research evaluates different methods for interacting with InfoVis systems, focusing on different kinds of visualizations and appropriate means for interaction. Notable papers in this area are \cite{research:interaction:tu, research:interaction:lam}

\paragraph{Geographic Visualization}
By applying geographic data to visualzations, data sets can be geographically weigthed, thus providing information about geographical distribution and local anomalies. Current research concentrates on generating this geographical weight. For this, data sets have to be mapped to geographical locations. The generation of geographic data, however, is non trivial and depends on the scenario. Notable papers in this area are \cite{research:geography:dykes, research:geography:fisher}

\paragraph{Reusability of visualizations}
Visualization systems typically deal with a predefined set of scenarios. Applying them beyond those scenarios is not feasible. One topic of research is to identify means on how visualizations and InfoVis systems can be reused for other scenarios in order to reduce development and implementation effort. A notable paper in this area is \cite{research:reuse:humphrey}

\paragraph{InfoVis for ranking and ordering data}
One very powerfull application of information visualization is to use it for sorting data. Especiall web search engines can profit greatly from the appliance of InfoVis technology. A visual ordering and sorting allows users to quickly identify the data entries that are most relevant to their query. Furthermore, it provides a sophisticated approach for clustering data. Notable papers in this area are \cite{research:ranking:kidwell, research:ranking:freiler}

\paragraph{Multi-dimensional visualization}
When constructing a visualization, it is important to evaluate the different dimensions both of the provided data set and of the constructed visualization. Current research trys to find out how the dimensions data sets can be mapped to those of the visualization, and in which cases the number of dimensions should be reduced or increased. It furthermore concentrates on the presentation on multi-dimenional visualizations. Notable papers in this area are \cite{research:dimensions:hanson, research:dimensions:yang, research:dimensions:riazati, research:dimensions:peng, research:dimensions:artero}

\paragraph{Distributed and/or collaborative visualization}
When working with larger data sets or complex structures, the process of recognition can be distributed amongst a set of human clients. These clients need to communicate and cooperate with one another thus increasing the overall process of recognition to an optimum. It is a current issue of research, how different clients can interoperate with oneanother under the prospect of efficiency optimization. Furthermore, classical InfoVis systems perform all preprocessing and transformation at a single point. Since in a collaborative scenario information is accessed from multiple clients, and since those clients typically require only a subset of the data pool, it is possible to distribute the computation amongst those clients thus resolving the need for a single high-end server and minimizing the costs of computation. Notable papers in this area are \cite{research:collaborative:zhao, research:collaborative:anupam, research:collaborative:heer, research:collaborative:johnson, research:collaborative:balakrishnan, research:collaborative:bajaj}

\paragraph{Scalability of information visualization}
Hardware capabilities such as resolution and display size are rapidly increasing. While this allows InfoVis systems to display an equally bigger the amount of information, the human capablities for recognition are limited. It is a current topic of research to identify the means, potentials, and limitations to scalable InfoVis systems in order to exploit the new capabilities as effective as possible. Notable papers in this area are \cite{research:scalability:yost, research:scalability:ball}

\paragraph{Text document visualization}
Especially large text documents often require a considerable amount of work in order to fully comprehend the content, its logical structure, and consequences. Furthermore, putting text documents into relationships with oneanother is both difficult. Since there is no scientific approach on capturing the concept and content of a text document, connections must often be made on instinct of the reader. By applying information visualization, it is hoped to gain knowledge about the structure of single documents and the relationship between different documents more easily. Notable papers in this area are \cite{research:text:weber, research:text:huang, research:text:rohrer, research:text:wattenberg}