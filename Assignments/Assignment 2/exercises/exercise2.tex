\section{Briefly summarize what the ``opponent process theory of colour'' means for selecting color values for use in information visualization systems}

The \emph{opponent process theory of colour} postulates that the signals received by the three cones in the human eye are encoded into three different color channels: red-green, blue-yellow, and black-white. These channels are aligned orthogonal to one another, forming a octahedron. The HSB color space is based on this octahedron. A specific color is described by its \textbf{H}ue, representing the angle of spin around the black-white axis, its \textbf{S}aturation, meaning the distance from this axis, and by its \textbf{B}rightness, being the position on that axis. \\
The three channels result in in six different basic colors placed on their far ends: red, blue, green, yellow, white and black. Most cultures recognize these colors as the primary colors identifying and naming them before any other color as shown in \cite{color:Berlin}. Since those colors are cross-culturally identified as the most dominant and relevant colors, they should be used in information visualization for identifying distinct data categories. Due to their distance on the channels, these colors are easy to identify and to separate from one another. This way, the distinction between the different data categories is alleviated. \\
When labeling continuous data, on the other hand, the different colors of the used spectrum must be perceptually ordered. This is necessary to ensure that all viewers can identify the order of the different colors correctly. By using a monotonous change on one or more of the color channels -- for example from blue to yellow, from red to green, or from blue to green -- this can be achieved.