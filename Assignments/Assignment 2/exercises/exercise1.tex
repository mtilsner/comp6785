\section{Chose two topics from the list you provided in Assignment 1. Find two recent peer-reviewed research papers for each topic, and provide a brief summary of the primary contributions of each paper}

\subsection{Topic 1: Visualizing Network Traffic for Administrators}
\subsubsection{Description}
As distributed software systems become more and more popular, networks are increasingly used in order to deploy and provide applications. These networks require special attention by administrators assuring the ongoing operation of the network and maintaining a level of security enabling the clients to operate undisturbed and uncompromised. The visualization of ongoing network traffic can greatly improve this administration by visualizing life data thus simplifing the cognition of threats such as attacks or security holes. Other notable papers in this area are \cite{network:other:keim, network:other:iliofotou, network:other:komlodi, network:other:lemal, network:other:becker, network:other:becker2, network:other:yoshida, network:other:oline}.

\subsubsection{Paper 1: \cite{network:ball}}
This paper introduces the Visual Information Security Utility for Administration Live (VISUAL): a toll that visualizes patterns in the communication between external hosts and internal clients. By looking at these patterns, an administrator can easily acquire insights about network activity. The entire network traffic can hereby be monitored simultaneously, allowing the administrator both to get a good grasp of the overall situation. Applying Shneiderman's directive \cite{visualization:shneiderman} to provide first an overview, then zooming in and filtering, and finally providing details, the administrator can then zoom in to acquire additional information such as ports, IP addresses, or the relative workload. This way, specific points of interest, such as increased communication with critical hosts, can be identified and examined.

\subsubsection{Paper 2: \cite{network:abdullah}}
This paper concentrates on detecting network intrusion by focusing on port activity. It suggests displaying the aggregated port activity in a stacked histogram to provide the administrator with an overview of the current situation. By again applying Shneidermans's overview-zoom in and filter-details directive from \cite{visualization:shneiderman}, the administrator can afterwards zoom into those histograms to acquire more information about certain points of interest. The paper furthermore describes a number of intrusion attacks and demonstrates how the developed visualization can be utilized to detect them.


\subsection{Topic 2: Visualizing search results}
\subsubsection{Description}
Especially when retrieving information, search engines are required to order and sort the list of documents matching the query in order to equip the user with the means to easily find whatever he is looking for. Unfortunately, this list is commonly provided in a format that is hard to comprehend for the user. While programs and algorithms have no trouble comparing and evaluating a list of several thousand entries, human users are often overburdened. Especially when searching for words with different domains such as \emph{Java}\footnote{Java is both a programming language and an island}, common sorting algorithms often fail to provide the user with the means to grasp the results. Visualization can help the user to comprehend the result list and to understand the relationship of individual result items. This way, the user can identify the results that are relevent for him more easily. Other notable papers in this area are \cite{search:other:lijn, search:other:rauber, search:other:benjamin, search:other:einsfeld, search:other:weippl, search:other:nowell, search:other:alonso, search:other:mukherje2, search:other:paulovich, search:other:konchady, search:other:tvarozek, search:other:hoeber, search:other:mukherjea, search:other:wong, search:other:russinov}.

\subsubsection{Paper 1: \cite{search:sebrechts}}
This paper compares textual, two-dimensional, and three-dimensional visualization for displaying search results. It discovers that finding a specific target is much easier in a textual presentation, while clustering and classification of documents benefit from 2D and 3D presentations. Furthermore, even though the three-dimensional application did have the worst performance, it also had the best learning rate, indicating that a trained user might be able to work with it very efficiently. In fact, test subject with greater computer skills responded more rapidly when working with the tree-dimensional presentation than when limited to two dimensions.

\subsubsection{Paper 2: \cite{search:zaina}}
This paper concentrates on putting the results of a web search into relationship with one another. It provides a tool that retrieves the content of the first 50kB for each search result. By comparing the content of each result, similarities are discovered and the documents are put into relationship with one another. The result is a two-dimensional graph displaying the results as a map. Each result is presented by an icon with its size corresponding to the documents position in the result list. The icons are placed so that their distance corresponds to their similarity. Furthermore, similar documents are connected by threads. A table displays all details of the documents and by selecting an icon the downloaded content is displayed thus allowing a quick preview of the document. Especially the table was often used by the users, suggesting an increased emphasis on details. The lack of following Shneiderman's directive from above might have provided a larger acceptance.